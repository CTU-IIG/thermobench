\documentclass[a4paper,UKenglish]{oasics-v2016}
%This is a template for producing OASIcs articles.
%See oasics-manual.pdf for further information.
%for A4 paper format use option "a4paper", for US-letter use option "letterpaper"
%for british hyphenation rules use option "UKenglish", for american hyphenation rules use option "USenglish"

\usepackage{microtype}%if unwanted, comment out or use option "draft"

\usepackage{booktabs}

\usepackage{array}
\usepackage{tikz}
\usepackage{pgfplots}

%\graphicspath{{./graphics/}}%helpful if your graphic files are in another directory

\bibliographystyle{plain}% the recommended bibstyle


\newcommand{\todo}[1]{{\emph{TODO: #1}}}
\newcommand{\martin}[1]{{\color{blue} Martin: #1}}
\newcommand{\abcdef}[1]{{\color{red} Author2: #1}}

\newcommand{\benchcount}{54 }

\newcommand{\code}[1]{{\small{\texttt{#1}}}}

% remove the comment for the submission
\renewcommand{\todo}[1]{}
\renewcommand{\martin}[1]{}


% fatter TT font
\renewcommand*\ttdefault{txtt}

\newcolumntype{L}[1]{>{\raggedright\let\newline\\\arraybackslash\hspace{0pt}}m{#1}}
\newcolumntype{C}[1]{>{\centering\let\newline\\\arraybackslash\hspace{0pt}}m{#1}}
\newcolumntype{R}[1]{>{\raggedleft\let\newline\\\arraybackslash\hspace{0pt}}m{#1}}

% Author macros::begin %%%%%%%%%%%%%%%%%%%%%%%%%%%%%%%%%%%%%%%%%%%%%%%%
\title{TACLeBench: A Benchmark Collection to Support Worst-Case Execution
Time Research\footnote{This work was partially supported by COST Action
IC1202 Timing Analysis on Code-Level (TACLe).}}
%\titlerunning{TACLeBench} %optional, in case that the title is too long; the running title should fit into the top page column

% Heiko first, rest in alphabetical order
\author[1]{Heiko Falk}
\author[2]{Sebastian Altmeyer}
\author[3]{Peter Hellinckx}
\author[4]{Bj{\"o}rn Lisper}
\author[5]{Wolfgang Puffitsch}
\author[6]{Christine Rochange}
\author[5]{Martin Schoeberl}
\author[5]{Rasmus Bo S{\o}rensen}
\author[7]{Peter W{\"a}gemann}
\author[8]{Simon Wegener}
\affil[1]{Hamburg University of Technology, Institute of Embedded Systems, Germany \\
  \texttt{Heiko.Falk@tuhh.de}}
\affil[2]{University of Amsterdam, The Netherlands\\
  \texttt{altmeyer@uva.nl}}
\affil[3]{University of Antwerp, iMinds, Belgium\\
  \texttt{peter.hellinckx@uantwerpen.be}}
\affil[4]{M{\"a}lardalen University, School of Innovation, Design, and Engineering, Sweden\\
  \texttt{bjorn.lisper@mdh.se}}
\affil[5]{Technical University of Denmark, Department of Applied Mathematics and Computer Science, Denmark\\
  \texttt{\{wopu,masca,rboso\}@dtu.dk}}
\affil[6]{University of Toulouse, France\\
  \texttt{rochange@irit.fr}}
\affil[7]{Friedrich-Alexander University Erlangen-N{\"u}rnberg, Germany\\
  \texttt{waegemann@cs.fau.de}}
\affil[8]{AbsInt Angewandte Informatik GmbH, Germany\\
  \texttt{wegener@absint.com}}
  \authorrunning{H.~Falk et al.}%mandatory. First: Use abbreviated first/middle names. Second (only in severe cases): Use first author plus 'et. al.'

\Copyright{Heiko Falk et al.}%mandatory. OASIcs license is "CC-BY";  http://creativecommons.org/licenses/by/3.0/

\subjclass{C.3 SPECIAL-PURPOSE AND APPLICATION-BASED SYSTEMS, Real-time and embedded systems}
%\subjclass{Dummy classification -- please refer to \url{http://www.acm.org/about/class/ccs98-html}}% mandatory: Please choose ACM 1998 classifications from http://www.acm.org/about/class/ccs98-html . E.g., cite as "F.1.1 Models of Computation".
\keywords{Benchmark, WCET analysis, real-time systems}
% Author macros::end %%%%%%%%%%%%%%%%%%%%%%%%%%%%%%%%%%%%%%%%%%%%%%%%%

%Editor-only macros:: begin (do not touch as author)%%%%%%%%%%%%%%%%%%%%%%%%%%%%%%%%%%
\EventEditors{Martin Schoeberl}
\EventNoEds{1}
\EventLongTitle{16th International Workshop on Worst-Case Execution Time
Analysis (WCET 2016)}
\EventShortTitle{WCET 2016}
\EventAcronym{WCET}
\EventYear{2016}
\EventDate{July 5, 2016}
\EventLocation{Toulouse, France}
\EventLogo{}
\SeriesVolume{55}
\ArticleNo{2}
% Editor-only macros::end %%%%%%%%%%%%%%%%%%%%%%%%%%%%%%%%%%%%%%%%%%%%%%%

\begin{document}

\maketitle

\begin{abstract}
Engineering related research, such as research on worst-case execution time,
uses experimentation to evaluate ideas. For these experiments we need
example programs. Furthermore, to make the research experimentation
repeatable those programs shall be made publicly available.

We collected open-source programs, adapted them to a common coding
style, and provide the collection in open-source. The benchmark collection
is called TACLeBench and is available from GitHub in version 2.0
at the publication date of this paper. One of the main features of TACLeBench
is that all programs are self-contained without any dependencies on standard
libraries or an operating system.
\end{abstract}

\section{Introduction}
\label{sec:intro}

Good, realistic benchmark suites are essential for the evaluation and comparison
of worst-case execution time (WCET) analysis, compiler, and computer architecture techniques.
TACLeBench provides a freely available and comprehensive benchmark suite
for timing analysis and related research topics.
%, featuring complex multi-core benchmarks in the near future.
TACLeBench will be continuously extended by novel benchmarks,
especially by parallel multi-task/multi-core benchmarks.
The extension of TACLeBench will be carefully managed with snapshots
and versioning so that it is clear which code has been used in a research
experiment.
The overall goal is to establish TACLeBench as the standard benchmarking
suite for WCET analysis, WCET oriented compiler and computer
architecture research worldwide.

TACLeBench is a collection of currently \benchcount benchmark programs
from several different research groups and tool vendors around the world.
These benchmarks are provided as ISO C99 source codes.
The source codes are 100\% self-contained; no dependencies to system-specific
header files via \code{\#include} directives or an operating system exist.
All input data is part of the C source code.
Potentially used functions from math libraries are also provided in the form of C source code.
This makes the TACLeBench collection useful for general embedded/barebone
systems where no standard library is available.

Furthermore, almost all benchmarks are processor-independent and can
be compiled and evaluated for any kind of target processor.
The only exception is PapaBench that uses architecture-dependent
I/O addresses and currently supports Atmel AVR processors only.

Since TACLeBench addresses the needs imposed by timing analysis tools,
all benchmarks are completely annotated with flow facts.
These flow facts are directly incorporated into the C source codes using
Pragmas. % at a high level of abstraction.
TACLeBench distinguishes between so-called flow restrictions, loop bounds, and entry points.
% The following details shall go into its own section on a future publication with more space
%Flow restrictions allow to express the execution count of one statement in relation to the execution count of other statements.
%This annotation methods allows to express any kinds of linear dependencies between execution counts of statements.
%Flow restrictions are expressed in terms of inequations, where each side of an inequation is a weighted sum of execution counts of statements, and where the two sides of an inequation are compared by less-than-equal, equal, or greater-than-equal operators.
%For example, flow restrictions can be used to express precise upper execution bounds for irregularly nested loops (e.g., triangular loops), recursions, multi-entry loops etc.
Besides flow restrictions, TACLeBench also uses loop bound flow facts, which simplify the annotation of regular loops.
Loop bounds provide an upper and a lower bound for the number of iterations of the annotated loop.
%
Finally, TACLeBench uses entry point annotations that denote points in a program's control flow graph where the control flow may start.
Typically, this is the ``main'' function of a program, but in a (possibly interrupt-driven) multi-task system, there may be multiple entry points in a single set of source files.
These entry points may even share some common code.
In order to mark such task entries, each function of a multi-task application where a task begins can be marked as an entry point.
% For example, the following source code snippet states that the shown loop
% iterates between 50 and 100 times, depending on the data dependency
% via variable \code{maxIter}:
%
% \begin{verbatim}
%       _Pragma( "loopbound min 50 max 100" )
%
%       for ( i = 1; i <= maxIter; i++ )
%
%         Array[i] = i * fact * KNOWN_VALUE;
%
% \end{verbatim}
The complete specification of the used flow fact language can be found in~\cite{flowfacts},
which is part of the source distribution.


If you would like to share your  benchmarks with us,
feel free to contact Heiko Falk (or any coauthor of this paper) in order
to have your source codes included in TACLeBench.

The first version of TACLeBench (version 1.0, available
from\footnote{\url{http://www.tacle.eu/index.php/activities/taclebench}}), which was produced
by Heiko Falk, was a collection of 102 programs from several different
research groups. We keep this first version tagged with ``V1.0'' in the public
GitHub repository.\footnote{\url{https://github.com/tacle/tacle-bench}}
The version described in this paper is version 2.0 and tagged as such in the
repository.
The intention is to have the HEAD of the master branch being
the most recent, versioned snapshot of TACLeBench.
Future development and additions will be performed on a development
branch so that HEAD of master is always a consistent snapshot.

This paper is organized in 5 sections: The following section presents related work.
%Section~\ref{sec:related} presents related work.
Section~\ref{sec:collect} presents the benchmark collection, its classification, and the updates
to make them useful.
Section~\ref{sec:eval} evaluates the benchmark collection.
Section~\ref{sec:conclusion} concludes.

\section{Related Work}
\label{sec:related}

The M{\"a}lardalen WCET benchmarks (MRTC)~\cite{wcet:bench:2012} is the first collection of programs especially
intended for benchmarking WCET analysis tools, with a focus on program flow analysis.
%This collection of C programs was an attempt to organize a number of codes, which had up to then been floating around
%in the research community, into a repository with a more systematic treatment of meta-data and easy web access.
It was collected from several sources in 2005, and has since then been
used in many WCET research projects as well as for the WCET Tool Challenge 2006~\cite{Gustafsson:ISOLA2006}.
A subset of the M{\"a}lardalen benchmarks has even been translated to
Java~\cite{jop:volta:rtas2008}. Most benchmarks are relatively small, except
two C programs that have been generated from tools.
The benchmarks also contain all input data. This effectively turns them into single-path programs, which
makes them less suitable for evaluating tools that can handle multi-path codes.
%To remedy that,
%some of the benchmarks are equipped with suggested input ranges that override the fixed
%input data in the code. The user is however then responsible to use the proper settings of the tool,
%or to create an appropriate test harness, such that the programs are analyzed with the specified input ranges
%rather than the fixed input data.
We include most of the benchmarks from the
M{\"a}lardalen WCET benchmark suite in TACLeBench.
We dropped benchmarks where the licensing terms are unknown
or even disallow distributing the source.

MiBench~\cite{MiBench} is a collection of benchmarks targeting the embedded
domain and providing them in open-source.
We include some of the MiBench benchmarks, especially
those where it was possible to include the input data with the C source.

DEBIE~\cite{debie} is a program derived from a satellite-mounted detector of micro-meteorids and space debris.
It was developed by Space Systems Finland Ltd and was converted by Tidorum Ltd into a portable benchmark for real-time applications.
DEBIE is a multi-task application that consists of 8, partially small, different tasks.
DEBIE is accompanied by a specification of valid input and output data and of required activation rates of the individual tasks.

PapaBench~\cite{papabench} has been derived from  Paparazzi,\footnote{\url{https://wiki.paparazziuav.org/}}
a project for unmanned aerial vehicles.  PapaBench includes
two software components that run on separate processors:
the \textit{fly-by-wire} part controls the flight while the \textit{autopilot} part controls the GPS and
executes the flight plan (which is decided offline). Both software parts  cumulate 13 tasks that are subject
to precedence constraints and 6 interrupt service routines.

The Embedded Microprocessor Benchmark Consortium (EEMBC)~\cite{eembc} provides a
benchmark suite dedicated to the evaluation of
the performance of embedded hardware and embedded software.
The benchmarks are divided into subset according to the target domain, e.g., the automotive domain, phones and tablets, but also big data and cloud computing.
To improve comparability between different systems, the consortium provides a
test-harness that allows to derive certifiable scores.
The test-harness, being a clear advantage in terms of comparability,
constitutes a hindrance in terms of portability and usability.
Whereas the TACLeBench has been designed to ease portability and to allow the immediate use of the benchmark with a large variety of tools and platforms, the EEMBC benchmarks are not stand-alone executable without the test-harness.
Furthermore, in stark contrast to TACLeBench, the EEMBC benchmarks are not published under an open-source license.
Instead, the benchmarks are behind a pay-wall, even for purely academic research.


JemBench~\cite{jembench} is a Java benchmark suite targeting
embedded Java platforms. JemBench only assumes the
availability of a CLDC API, the minimal configuration
defined for the J2ME. The core of the benchmark suite consists of
adapted real-world applications.
The benchmarks are structured in {\em micro}, {\em kernel}, {\em
application}, {\em parallel}, and {\em streaming} benchmarks.
Micro benchmarks are used to measure short bytecode sequences;
kernel benchmarks compute a computational kernel; and application
benchmarks are real-world programs restructured as standalone benchmarks.
Parallel and streaming benchmarks are intended to explore multicore
speedup.
%\martin{Maybe we should port some of those to C?}
Benedikt Huber ported one of the application benchmarks (Lift) to C
and we include it in TACLeBench.


\begin{table}[t!]
\centering
\caption{\label{tab:bench_kernel}TACLeBench kernel benchmarks}
\begin{tabular}{lL{5.5cm}rR{2.5cm}}
\toprule
Name & Description & Code Size & Origin\\
     &             &     (SLOC) & \\ \midrule
binarysearch & Binary search of 15 integers & 47 & SNU-RT \\
bitcount & Couting number of bits in an integer array & 164  & Bob Stout \& Auke Reitsma\\
bitonic &  Bitonic sorting network  & 52 & MiBench \\
bsort & Bubblesort program & 32 & MRTC \\
complex\_updates &  Multiply-add with complex vectors & 18 & DSPStone \\
countnegative &  Counts signes in a matrix  & 35 & MRTC \\
crc &  CRC computation on 40 bytes of data  & 66 & SNU-RT \\
fac &  Factorial function & 21 & MRTC \\
fft &  1024-point FFT, 13 bits per twiddle  & 78  & DSPStone \\
filterbank & Filter bank for multirate signals  & 75 & StreamIt \\
fir2dim &  2-dimensional FIR filter convolution & 75 & DSPStone \\
iir &  Biquad IIR 4 sections filter & 27 & DSPStone \\
insertsort & Insertion sort & 35 & SNU-RT \\
jfdctint & Discrete-cosine transformation on a 8x8 pixel block  & 123  & SNU-RT \\
lms &  LMS adaptive signal enhancement  & 51  & SNU-RT \\
ludcmp & LU decomposition & 68 & SNU-RT \\
matrix1 &  Generic matrix multiplication  & 28 & DSPStone \\
md5 &  Message digest algorithm & 344  & NetBench \\
minver & Floating point matrix inversion  & 141  & SNU-RT \\
pm & Pattern match kernel & 484  & HPEC \\
prime &  Prime number test  & 41 & MRTC \\
quicksort &  Quick sort of strings and  vectors & 992 & MiBench \\
recursion &  Artificial recursive code  & 18 & MRTC \\
sha &  NIST secure hash algorithm & 382  & MiBench \\
st & Statistics calculations & 90  & MRTC \\
\bottomrule
\end{tabular}
\end{table}

\section{The Benchmark Collection}
\label{sec:collect}

\subsection{Benchmark Sources and Classification}

The benchmarks included in TACLeBench are sourced from single sources and
benchmarks collections.
The benchmarks are: SNU-RT benchmark suite, MiBench embedded benchmark suite, M{\"a}lardalen Real-Time Research Center (MRTC) WCET benchmarks,
DSPStone from RWTH Aachen, StreamIt from MIT, NetBench from UCLA, MediaBench, and the HEPC challenge benchmark suite.
We have specified the origin of each benchmark in Tables~\ref{tab:bench_kernel}-\ref{tab:bench_app}.

%\todo{what did we collect. How many indirections in collections have there be?
%E.g., adpcm.c: (1) SOURCE : C Algorithms for Real-Time DSP by P. M. Embree,
%(2) SNU-RT Benchmark Suite, (3) MDH, and (4) TACLeBench. How have
%benchmarks  changed? Legal changes?}
%\todo{Rasmus}

%In this subsection, we provide an overview of the classes of benchmarks that are found in the TACLeBench suite and list the individual benchmarks.
As a measure of the size of each benchmark, we present the number of source lines of code (SLOC).
The SLOC count excludes input data arrays and the initialization code.
We used the Linux utility \code{sloccount} to measure the SLOC.
The benchmarks are divided into five classes: Kernel, sequential, application, test, and parallel.



The kernel benchmarks, listed in Table~\ref{tab:bench_kernel}, are synthetic benchmarks implementing small kernel functions; the size of the kernel benchmarks is in the range of 18 to 992 SLOC.
%
\begin{table}[t]
\centering
\caption{\label{tab:bench_sequential}TACLeBench sequential benchmarks}
\begin{tabular}{lL{5.5cm}rR{2.5cm}}
\toprule
Name & Description & Code Size & Origin\\
   & &  (SLOC) & \\ \midrule
adpcm\_dec & ADPCM decoder & 293 & SNU-RT \\
adpcm\_enc &  ADPCM encoder  & 316  & SNU-RT \\
ammunition & C compiler arithmetic stress test  & 2431 & Vladimir Makarov \\
anagram &  Word anagram computation & 2710  & Raymond Chen \\
audiobeam &  Audio beam former  & 833  & StreamIt \\
cjpeg\_transupp &  JPEG image transcoding routines  & 608 & MediaBench \\
cjpeg\_wrbmp & JPEG image bitmap writing code & 892 & Thomas G. Lane \\
dijkstra & All pairs shortest path  & 117  & MiBench \\
epic & Efficient pyramid image coder  & 451  & MediaBench \\
fmref &  Software FM radio with equalizer & 680  & StreamIt \\
g723\_enc &  CCITT G.723 encoder  & 480  & SUN Microsystems \\
gsm\_dec & GSM provisional standard decoder & 543 & MediaBench \\
gsm\_enc & GSM provisional standard encoder & 1491 & MediaBench \\
h264\_dec &  H.264 block decoding functions & 460 & MediaBench \\
huff\_dec &  Huffman decoding with a file source to decompress  & 183  & David Bourgin \\
huff\_enc & Huffman encoding with a file source to compress  & 325  & David Bourgin \\
mpeg2 &  MPEG2 motion estimation  & 1297 & MediaBench \\
ndes & Complex embedded code  & 260  & MRTC \\
petrinet & Petri net simulation & 500  & Friedhelm Stappert \\
rijndael\_dec & Rijndael AES decryption  & 820 & MiBench \\
rijndael\_enc & Rijndael AES encryption  & 734 & MiBench \\
statemate &  Statechart simulation of a car window lift control & 1038 & Friedhelm Stappert \\
susan &  MR image recognition algorithm & 1491 & MiBench \\
\bottomrule
\end{tabular}
\end{table}
%
The sequential benchmarks, listed in Table~\ref{tab:bench_sequential},
implement large functions block,
such as encoders and decoders, that are used in many embedded systems.
The size of the sequential benchmarks is in the range of 117 to 2710 SLOC.
The sequential benchmarks cover graph search, cryptographic algorithms, compression algorithms, etc.
%
\begin{table}
\centering
\caption{\label{tab:bench_test}TACLeBench test benchmarks}
\begin{tabular}{lL{5.5cm}rR{2.5cm}}
\toprule
Name & Description & Code Size & Origin\\
     &             &  (SLOC) & \\ \midrule
cover &  Artificial code with lots of different control flow paths  & 620  & MRTC \\
duff & Duff's device  & 35 & MRTC \\
test3 &  Artificial WCET analysis stress test & 4235 & Universitaet des Saarlandes \\
\bottomrule
\end{tabular}
\end{table}
%
Three artificial test benchmarks, listed in Table~\ref{tab:bench_test}, are used
to stress test WCET analysis tools.
%
\begin{table}
\centering
\caption{\label{tab:bench_parallel}TACLeBench parallel benchmarks}
\begin{tabular}{lL{5cm}rrR{2cm}}
\toprule
Name & Description & \#Tasks &Code Size & Origin\\
     &             &         &(SLOC) & \\ \midrule
Debie&  DEBIE-1 instrument observing micro-meteoroids and small space debris &8 &  6615 & Tidorum Ltd \\
PapaBench &  UAV autopilot and fly-by-wire software & 10 & 6336 & Paparazzi \\
\bottomrule
\end{tabular}
\end{table}
%
The two parallel benchmarks, listed in Table~\ref{tab:bench_parallel}, are:
Debie and PapaBench.
These two benchmarks are comparable in size and in the number of tasks.

The application benchmarks, derived from real applications and provided with simulated input stimuli, are listed in Table~\ref{tab:bench_app}.
%
Lift is a lift controller that has been deployed at a
factory in Turkey. The hardware is based on a Java processor (JOP).
% in an
%FPGA device and an I/O print that was originally developed
%for the supervisory control and data acquisition device.
The controller has just a few inputs (command buttons and input
sensors for the height measurement) and a simple motor control.
The I/O devices are simulated in the benchmark.
The Java version of Lift is part of the Java benchmark suit JemBench~\cite{jembench}.
Benedikt Huber has translated lift to C.

powerwindow implements a controller for an electric window in a car.
Both the driver and the passenger are able to control the window by 
requesting the window to roll up or down.
In case an object is stuck between the window and the door frame,
the controller will move the window down to avoid damaging the object.

\begin{table}
\centering
\caption{\label{tab:bench_app}TACLeBench application benchmarks}
\begin{tabular}{lL{5cm}rrR{2cm}}
\toprule
Name & Description & \#Tasks & Code Size & Origin\\

     &             &         &     (SLOC) & \\ \midrule
lift & A lift controller & 1 & 361 & Martin Schoeberl\\
powerwindow & Distributed power window control & 4 & 2533 & CoSys-Lab \\
\bottomrule
\end{tabular}
\end{table}

\subsection{Issues with the Original Sources}


\todo{Show the issues with code snippets. Or the difference with the original one
and the updated one with patmos-clang -O3 and pasim.}

\todo{Show how compiler optimizations optimized away the original code and how we fixed this}

The original benchmarks include all input data or we added input data into the C source.
However, this effectively turns them into single-path programs.
This fact could be used by analysis tools to explore only this single path.
Another consequence of the fixed
input data, and that some programs do not provide
any return value, is that compilers with optimizations turned on can optimize
most of the code away.
However, to prohibit the unwanted compiler optimizations we changed
the way input data is represented in variables (made them \code{volatile}),
and made the return of \code{main} dependent on the benchmark calculation.

%Some benchmarks, which are part of V 1.0 of TACLeBench and are available
%for download in the Internet, have a license that does not allow open-source distribution.
%We removed those benchmarks from the original source.

\martin{Mmh, I think we need to fix this. The simplest form is dropping them
from the list. How many benchmarks depend on byte ordering?}
Some benchmarks contain target dependent code.
For example, PapaBench contains hardcoded I/O addresses.
Furthermore, a few benchmarks (e.g., rijndael) are byte order dependent
and there is no standard way in C to detect the byte order of a processor.
Finally, some benchmarks can be executed only once, either because
they rely on global initialization, or because they use \code{malloc} but not \code{free}.

%%%\todo{low-level access to device registers -- which benchmarks are involved?}

\subsection{Benchmark Updates}

The benchmarks have been rewritten to split the functions of
input data initialization, the benchmark itself, and computing a return
value depending on the output data. Moving the input data generation
into its own function resulted sometimes in movements from originally
stack allocated data into global data.
All function and variable names are prepended with the benchmark
name to provide unique names.
All loops have been annotated with loop bounds.
Moreover, several bugs have been fixed, and compiler warnings have been eliminated.
The benchmarks are now ISO C99 compliant.
Some benchmarks have been renamed. The original name of a benchmark can be found in the comment header of each benchmark.
The library functions used by some benchmarks have been moved to their own files.
All source files adhere to a common set of formatting rules that can be found in the git repository (\textit{doc/code\_formatting.txt}).

%%%\todo{What else has been done?}

Due to these changes, results obtained with the TACLeBench versions
of these benchmarks are not comparable with the original versions of
the benchmarks.


\subsection{Licenses}

An issue we encountered with several benchmarks was that the original
source did not include any licensing information. In absence of such
information, we had to assume that the copyright holder reserves all
rights. Wherever necessary, we contacted the copyright holders to
obtain the right to use, modify, and redistribute the benchmarks. In a
small number of cases we however discovered that the code was in fact
under a license that made the benchmark unusable; the respective
benchmarks were consequently dropped from the TACLeBench benchmark
suite. All benchmarks in the benchmark suite now contain licensing
information, such that future developments do not require tracking
down the original authors of the benchmark.





\subsection{Usage Recommendations}
\label{usage}

TACLeBench is released in source form. However, to report
results based on TACLeBench, the benchmarks shall not be changed.
Furthermore, the version of TACLeBench shall be included in any paper.
% Martin does not understand the following sentence.
% State the results for the entry points of the benchmarks.

If possible, use all benchmarks in your evaluation. One issue with subsetting
a benchmark collection is to \emph{selecting the most representative benchmarks}
to show an improvement. However, this introduces a bias in the result and
is considered scientific misconduct. If you really need to subset a benchmark collection you have
two possibilities: (1) use only one class of benchmarks, e.g., the sequential
benchmarks or (2) use a random selection, possibly generated by a program.

If some benchmarks have not been used, state the reason for exclusion
(e.g., ``the tool/target architecture does not support floating point numbers'').
% Martin does not like allowing benchmarks patching
%If a benchmark needs to be patched to run on a some architecture,
%publish what changes needed to be done (preferably as a patch).
\martin{I would really like to aim for a benchmark collection that does not
need any changes! Better drop some.}


\subsection{Known Uses of TACLeBench}

Although TACLeBench is a relative new collection of embedded benchmarks,
we can already list some usage of the collection in research projects.
This early adaption of TACLeBench is already a strong indication of
the need of such a benchmark collection.

\martin{This section might go away, as we have not enough pace.
However, it is good to collect the information here. We might need it
for a future journal version.}

\todo{Refs are missing form some usage.}

\begin{itemize}
%\item Static analysis: \todo{No usage up to now?}
\item Compiler optimizations: The origin of TACLeBench is the collection of free
programs used for evaluation of the wcc compiler~\cite{falk:wcc:wcet06}.
\item Measurement-based analysis: TACLeBench has been used to evaluate the continuous measurement-based WCET estimation approach presented
in~\cite{Dreyer2016}.
The different characteristics of the different benchmarks proved to be very useful to trigger edge cases in the analysis and led to various improvements of the prototype.
However, the evaluation for this approach happened before version 2.0 of TACLeBench has been finished,
which makes the results non-comparable to the final benchmark collection.
\martin{I am having a hard time understanding the (original) text on hybrid analysis.
What are the two layers? 
Anyway I try to rewrite and cutdown. We need to discuss this for the final version.}
\item Hybrid analysis: The hybrid model splits the code of tasks into basic blocks
and uses measurements to obtain instruction traces.
% Each basic block contains one path trace of instructions with a single set of inputs and a single set of outputs.  The  size  of  a  basic  block  varies  between  the  largest possible trace meeting the constraints of a single set of inputs and outputs over a single memory block, i.e. a unit of memory that is replaced by a cache replacement algorithm, to a single program instruction.
The  challenge  of  this  two-layer  hybrid  approach  is  tackling the computational complexity problems within the static analysis and accuracy  within  the  measurements based  layer.
% In  other  words,  a  proper  balance  needs  to  be found between the two layers of the hybrid model.
The TACLeBench is used in the COBRA-HPA (COde Behaviour fRAmework-Hybrid Program Analyser) framework that facilitates evaluation of the different approaches using different block sizes.
\martin{We need a link or paper for the above. I did not find anything just with Google.}
Furthermore, COBRA-TG (Taskset Generator) uses TACLeBench for schedulability analysis.
Different scheduling methodologies can be analyzed in a reproducible way using generated
tasksets based on specific application descriptions.
%It replaces the limited set of universal scheduling benchmarks and the wide variety of non-reproducible benchmark application with a methodology that enables for an unlimited but reproducible set of tasksets.
\item Hardware design: As the TACLeBench collection is self contained, it leads itself
as an easy to use benchmark collection to evaluate computer architecture design in the
embedded and real-time domain. We have used version 1.0 for the evaluation of the
stack cache design optimized for real-time systems~\cite{patmos:stackcache:rts}.

%\begin{itemize}
%%\item Static analysis: \todo{No usage up to now?}
%\item Compiler optimizations: The origin of TACLeBench is the collection of free
%programs used for evaluation of the wcc compiler.
%\item Measurement-based analysis: TACLeBench has been used to evaluate the continuous measurement-based WCET estimation approach presented
%in~\cite{Dreyer2015}.\footnote{A report of this evaluation has been submitted to the 2016 edition of the WCET workshop.}
%The different characteristics of the different benchmarks proved to be very useful to trigger edge cases in the analysis and led to various improvements of the prototype.
%Since the evaluation for this approach happened before version 2.0 of TACLeBench was finalized, some issues occurred that will be fixed in the final collection.
%This makes the results stated there non-comparable.
%Moreover, the best practices mentioned in Section~\ref{usage} have not been strictly followed.
%\item Hybrid analysis: The hybrid model splits the code of a set of tasks into so-
%called basic blocks. Each basic block contains one path trace of instructions with a single set of inputs and a single set of outputs.  The  size  of  a  basic  block  varies  between  the  largest possible trace meeting the constraints of a single set of inputs and outputs over a single memory block, i.e. a unit of memory that is replaced by a cache replacement algorithm, to a single program instruction.
%The  challenge  of  this  two-layer  hybrid  approach  is  tackling the computational complexity problems within the static analysis  layer  and  the  accuracy  within  the  measurements based  layer.  In  other  words,  a  proper  balance  needs  to  be found between the two layers of the hybrid model. The TACLeBench is used in the COBRA-HPA (COde Behaviour fRAmework-Hybrid Program Analyser) framework that facilitates evaluation of the different approaches using different block sizes.
%\item Schedulability analysis using TACLeBench is facilitated by the COBRA-TG (Taskset Generator). Different scheduling methodologies can be analyzed in a reproducible way using generated tasksets based on specific application descriptions. It replaces the limited set of universal scheduling benchmarks and the wide variety of non-reproducible benchmark application with a methodology that enables for an unlimited but reproducible set of tasksets.
%\item Hardware design: As the TACLeBench collection is self contained, it leads itself
%as an easy to use benchmark collection to evaluate computer architecture design in the
%embedded and real-time domain. We have used version 1.0 for the evaluation of the
%stack cache design optimized for real-time systems~\cite{patmos:stackcache:rts}.

% Based on taskset descriptions

\end{itemize}


\subsection{Source Access and Compiling the Benchmarks}

The benchmarks are hosted on GitHub at \url{https://github.com/tacle/tacle-bench}.
Each benchmark is in its own folder and can simply be compiled with your favorite
C compiler with following command:

\begin{verbatim}
  cc/gcc/clang *.c
\end{verbatim}


\section{Evaluation and Sanity Checks}
\label{sec:eval}
To evaluate the complexity of the benchmarks and their resilience against 
compiler optimizations, we executed each benchmark on \texttt{pasim}, a 
cycle-accurate simulator of the Patmos architecture~\cite{t-crest:2015}. Each 
benchmark was compiled using \texttt{patmos-clang} with activated compiler 
optimizations~(i.e., -O2). The results of this evaluation are summarized in 
Figure~\ref{fig:execution-times}. All execution traces start at the function 
marked as entry point by a pragma directive in the source code of the benchmark.
The execution times range from 305\,cycles~(binarysearch) up to 
1,658,333,567\,cycles~(test3).
%TODO: we should rename this artificial WCET analysis stress test
To put this into relation, the benchmark program test3 runs approximately 
for 21\,seconds on the Patmos platform assuming a CPU Frequency of 80\,Mhz.
%TODO: can we easily make this calculation?
% bs.c from M{\"a}lardalen is completely optimized to a 
% single return statement (in clang, not in patmos-clang).
% In patmos-clang bs requires only 26 cycles (starting) from main.
From this evaluation we make the main observation that 
TACLeBench consists of both short- and long running benchmarks with a huge variety in execution time. 
No benchmark in the suite is optimized to a single return statement.

\begin{figure}[t]
  \def\resultfile{eval/wcet.csv}
  %
\makeatletter
\pgfplotsset{
    /pgfplots/flexible xticklabels from table/.code n args={3}{%
        \pgfplotstableread[#3]{#1}\coordinate@table
        \pgfplotstablegetcolumn{#2}\of{\coordinate@table}\to\pgfplots@xticklabels
        \let\pgfplots@xticklabel=\pgfplots@user@ticklabel@list@x
    }
}
\makeatother
%
\begin{tikzpicture}
\begin{axis}[
  ybar,
  xtick=data,
  ymin=0,
  xmin=-1,
  xmax=56,
  ymax=2000000000,
  xtick pos=left,
  ytick pos=left,
  font=\tiny,
  width=0.99\textwidth,
  height=0.4\textwidth,
  bar width=4pt,
  axis y line*=left,
  ylabel={\# Cycles},
  ylabel style={anchor=south, yshift=-20},
  flexible xticklabels from table={\resultfile}{benchmark}{col sep=space},
  xticklabel style={rotate=50,anchor=east,xshift=1},
  ymode=log,
]
%
\addplot [%
  draw=black, 
  fill=gray!5!blue!70, %
  top color=gray!5!blue!20, bottom color=gray!5!blue!60,
  ]
  table [x=position,
         y=pasim,
] {\resultfile};
\end{axis}
\end{tikzpicture}
%
  \caption{Execution times of programs in TACLeBench range from 305 cycles up to more than 1,600,000,000 cycles on the Patmos architecture.}
  \label{fig:execution-times}
\end{figure}

Different members of the TACLe COST action changed the code.
Such a collaborative procedure is inherently error-prone.
Human errors can occur and the original sources were often faulty to start with.
The distributive work on the benchmarks led to an additional source of error detection:
%Different system configurations can lead to different behaviors, and 
a benchmark behaving well under system configuration A may be faulty under configuration B.
To ensure the quality of the benchmarks and to improve portability, we have thus implemented automatic sanity checks.

The code quality
% and conformity with the code formatting rules
is validated centrally and the latest results can be viewed online.\footnote{\url{https://www4.cs.fau.de/Research/TACLeBench}}
For this sanity check, all benchmarks are compiled using \textit{gcc}, \textit{g$++$} and \textit{clang}, executed and the return values are checked against the expected value.
Clang's static analyzer\footnote{\url{http://clang-analyzer.llvm.org/}} is used to further reveal programming bugs, such as out-of-bounds errors, and also to validate the compliance of the code with the code-formatting rules.

The portability has been checked via a shell-script (\code{checkBenchmark.sh}) that is now part of the TACLeBench repository.
\martin{Should check again if it executes on a Mac.}
The script allows to quickly identifying incompatibilities of the benchmarks with specific operating systems, compilers, and system configurations.
Even though full coverage of all system configurations can never be achieved, we were able to cover most of the common operating systems and compilers.

% In order to improve the code quality of the TACLeBench suite, we used different
% compilers (e.g., clang, gcc) to identify shortcomings in the code of each benchmark in an
% automated way. Additionally, we exploited Clang's static
% analyzer\footnote{\url{http://clang-analyzer.llvm.org/}} to further reveal
% programming bugs in the benchmark suite. We extended this sanity checker with additional
% TACLeBench-specific passes to identify if all benchmark are compliant to the code-formatting rules.

\todo{Comparison compiler warning in TACLeBench v1 with v2, problem: expint was in 1.0, but is no
longer in 2.0 and powerwindow was added for 2.0}

%Possible Options for the Evaluation:
%
%\begin{itemize}
%\item Compile for an embedded processor and measure execution time
%\item Compile for Patmos and use pasim for execution cycles
%\item 
%\item Complexity numbers with bar charts on execution time and maybe tools
%\item Use some WCET tools
%\end{itemize}


\section{Conclusion}
\label{sec:conclusion}

Research in the field of embedded real-time systems needs benchmarks to
evaluate research ideas. TACLeBench provides an open source collection of
\benchcount benchmarks. As all benchmarks are self-contained, they are easy to
use in systems that are lacking the standard library or an operating systems.
%
This paper reports on the TACLeBench version 2.0.
We intend to grow the collection of programs and coordinate the
release of benchmark versions.
We further welcome contributions to the benchmark collection.



\subparagraph*{Acknowledgements}

We want to thank Bendikt Huber for porting the Lift benchmark from Java to C.
We want to thank Niklas Holsti from Tidorum Ltd for contributing DEBIE in open-source.
% Order according to contribution count on GitHub: https://github.com/tacle/tacle-bench/graphs/contributors
% But Simon is not listed there! Maybe we are missing some people?
We want to thank Yorick De Bock, Florian Kluge, J{\"o}rg Mische, and Haoxuan Li for helping in restructuring the benchmarks.


\bibliography{taclebench}


\end{document}
