\documentclass[a4paper,UKenglish]{oasics}
%This is a template for producing OASIcs articles.
%See oasics-manual.pdf for further information.
%for A4 paper format use option "a4paper", for US-letter use option "letterpaper"
%for british hyphenation rules use option "UKenglish", for american hyphenation rules use option "USenglish"

\usepackage{microtype}%if unwanted, comment out or use option "draft"

%\graphicspath{{./graphics/}}%helpful if your graphic files are in another directory

\bibliographystyle{plain}% the recommended bibstyle


\newcommand{\todo}[1]{{\emph{TODO: #1}}}
\newcommand{\martin}[1]{{\color{blue} Martin: #1}}
\newcommand{\abcdef}[1]{{\color{red} Author2: #1}}

\newcommand{\code}[1]{{\small{\texttt{#1}}}}

% fatter TT font
\renewcommand*\ttdefault{txtt}

% Author macros::begin %%%%%%%%%%%%%%%%%%%%%%%%%%%%%%%%%%%%%%%%%%%%%%%%
\title{TACLeBench -- Title is TBD\footnote{This work was partially supported TACLe. \emph{add the right phrase}}}
%\titlerunning{TACLeBench} %optional, in case that the title is too long; the running title should fit into the top page column

\author[1]{Heiko Falk}
\author[2]{Wolfgang Puffitsch}
\author[2]{Martin Schoeberl}
\author[2]{Add yourself in alphabetical order, but keep Heiko first.}
\affil[1]{Dummy University Computing Laboratory\\
  Address, Country\\
  \texttt{open@dummyuni.org}}
\affil[2]{Technical University of Denmark, Department of Applied Mathematics and Computer Science, Denmark\\
  \texttt{\{wopu, masca\}@dtu.dk}}
\authorrunning{H.~Falk et al.}%mandatory. First: Use abbreviated first/middle names. Second (only in severe cases): Use first author plus 'et. al.'

\Copyright{Heiko Falk et al.}%mandatory. OASIcs license is "CC-BY";  http://creativecommons.org/licenses/by/3.0/

\subjclass{Dummy classification -- please refer to \url{http://www.acm.org/about/class/ccs98-html}}% mandatory: Please choose ACM 1998 classifications from http://www.acm.org/about/class/ccs98-html . E.g., cite as "F.1.1 Models of Computation". 
\keywords{Dummy keywords -- please provide 1--5 keywords}% mandatory: Please provide 1-5 keywords
% Author macros::end %%%%%%%%%%%%%%%%%%%%%%%%%%%%%%%%%%%%%%%%%%%%%%%%%

%Editor-only macros:: begin (do not touch as author)%%%%%%%%%%%%%%%%%%%%%%%%%%%%%%%%%%
\serieslogo{}%please provide filename (without suffix)
\volumeinfo%(easychair interface)
  {Martin Schoeberl}% editors
  {2}% number of editors: 1, 2, ....
  {Conference/workshop/symposium title on which this volume is based on}% event
  {1}% volume
  {1}% issue
  {1}% starting page number
\EventShortName{}
\DOI{10.4230/OASIcs.xxx.yyy.p}% to be completed by the volume editor
% Editor-only macros::end %%%%%%%%%%%%%%%%%%%%%%%%%%%%%%%%%%%%%%%%%%%%%%%

\begin{document}

\maketitle

\begin{abstract}
We need benchmarks that fit the WCET and embedded community needs,
 \end{abstract}

\section{Notes}

\todo{Just internal notes to collect bullet points on ideas.}

\begin{itemize}
\item On issue with original benchmarks optimized away - we can now check with Patmos easily
\item Working on getting the license correct - issue with code from unknown source
\item Useful in general for embedded systems/barebone systems where less libraries are available
\item Usage of the benchmarks (version, no subsetting, no source change)
\item How to contribute
\end{itemize}

\section{Introduction}
\label{sec:intro}

\todo{A brief introduction what the paper is about. It shall include briefly the
main contributions and findings. The contributions can be bullet listed.}

This paper... \todo{purpose statement, latest in 4th paragraph}

\todo{We are aiming on self-contained benchmarks, that do not need any
operating system services, including file IO. Therefore,...}

\todo{Talk about the TACLe COST action and the subgroup around the benchmark
collection. And plans how to further develop and maintain the collection.}

\todo{Talk about usage of the benchmarks: WCET tools, compiler, hardware
architecture,...}

The first version of TACLeBench (version 1.0, available
from\footnote{\url{http://www.tacle.eu/index.php/activities/taclebench}}), which was produced
by Heiko Falk, was a collection of 102 benchmark programs from several different
research groups. We keep this first version tagged with ``V1.0'' in the public
GitHub repository.\footnote{\url{https://github.com/tacle/tacle-bench}}
The version described in this paper is version 2.0 and tagged as such in the
repository.

The contributions of this paper are: (1) ... (2) ...

This paper is organized in N sections: The following section presents related work.
Section~\ref{sec:related} presents related work.
Section~\ref{sec:collect} presents the benchmark collection, its classification, and the updates
to make them useful.
Section~\ref{sec:eval} evaluates the benchmark collection on...
Section~\ref{sec:conclusion} concludes.

\section{Related Work}
\label{sec:related}

\todo{Show that you know the field. All related work shall be put
into context or contrast to our current work.}

The M{\"a}lardalen WCET benchmarks is the first collection of programs especially
intended for benchmarking WCET analysis tools~\cite{wcet:bench:2012}.
This collection of C programs was collected from several sources in 2005 and
was used in many WCET research projects and for the WCET Tool Challenge 2006.
A subset of the M{\"a}lardalen benchmarks has even been translated to
Java~\cite{jop:volta:rtas2008}. Most benchmarks are relative small, except
two C programs that have been generated from tools. The benchmarks also
contain all input data. As this input data is fixed and some programs do not provide
any return value, good compilers with optimization turned on can optimize
most of the code away. We include most of the benchmarks from the
 M{\"a}lardalen WCET benchmark suite in TACLeBench. However, we changed
 the way input data is represented in variables (make them \code{volatile})
 and make the return of \code{main} dependent on the benchmark calculation.
 Furthermore, we dropped benchmarks where this licensing terms are unknown
 or even disallow distributing the source.

EEMBC

MiBench~\cite{MiBench} We include some of the MiBench benchmarks, especially
those where it was possible to include the input data with the C source.

Debie~\cite{debie}

PapaBench~\cite{papabench}

General purpose benchmarking, with a focus on open source and embedded systems.

JemBench.

\section{The Benchmark Collection}
\label{sec:collect}


\subsection{Benchmark Sources}

\todo{what did we collect. How many indirections in collections have there be?
E.g., adpcm.c: (1) SOURCE : C Algorithms for Real-Time DSP by P. M. Embree,
(2) SNU-RT Benchmark Suite, (3) MDH, and (4) TACLeBench. How have
benchmarks  changed? Legal changes?}

\subsection{Classification}

\todo{Talk about kernel, medium size, and application benchmarks and their usage.}
\begin{itemize}
\item Static analysis
\item Measurement based analysis
\item Schedulability analysis using taskset generator
\end{itemize}

\subsection{Issues with the Original Sources}


\todo{Show the issues with code snippets}

\begin{itemize}
\item Compiler optimization
\item byte order
\item low-level access to device registers
\item Some benchmarks, which are available for download in the Internet have a license that
does not allow open-source distribution. We removed those benchmarks from the original source.
\end{itemize}

\subsection{Benchmark Changes}



\begin{itemize}
\item unique names for functions and global variables
\item Sometimes movements of stack allocated variables into global
\item splitting of initialization and computation code into two functions
\item Loop bounds added
\end{itemize}

\subsection{Usage Recommandations}

\todo{Mmh shall we have this?}

\todo{can we include a paragraph about the taskset generator for schedulability analysis}

\begin{itemize}
\item Use a stable version. At the time of this writing we should have V 2.0
\item Do not edit the benchmarks
\item Use all benchmarks in your evaluation. Don't introduce a bias in your evaluation by
\emph{selecting the most representative benchmarks} for your improvement
\end{itemize}

\subsection{Licenses}

An issue we encountered with several benchmarks was that the original
source did not include any licensing information. In absence of such
information, we had to assume that the copyright holder reserves all
rights. Wherever necessary, we contacted the copyright holders to
obtain the right to use, modify, and redistribute the benchmarks. In a
small number of cases we however discovered that the code was in fact
under a license that made the benchmark unusable; the respective
benchmarks were consequently dropped from the TACLeBench benchmark
suite. All benchmarks in the benchmark suite now contain licensing
information, such that future developments do not require tracking
down the original authors of the benchmark.

\section{Evaluation}
\label{sec:eval}

\todo{Computer engineering is a constructive science. We build stuff and we measure.
Therefore, there shall always be an evaluation section.}

Possible Options for the Evaluation:

\begin{itemize}
\item Compile for an embedded processor and measure execution time
\item Compile for Patmos and use pasim for execution cycles
\item Show how compiler optimizations optimized away the original code and how we fixed this
\item Complexity numbers with bar charts on execution time and maybe tools
\item Use some WCET tools
\end{itemize}


\section{Conclusion}
\label{sec:conclusion}

\todo{Rephrase what this paper is about and list the main contributions and results.}

\subparagraph*{Acknowledgements}

\todo{TACLe COST action phrase}

We want to thank \dots for helping in restructuring the benchmarks.
We want to thank Bendikt Huber for porting the Lift benchmark from Java to C.

\todo{Ack Niclas for contributing DEBIE in open-source.}

\appendix
\section{Source Access and Compiling the Benchmarks}


\bibliography{taclebench}


\end{document}
